\chapter{Introduction}
\label{cha:intro}

In the first two parts of this chapter a brief overview will explain the most known databases technologies while in the last part NoSQL databases will be introduced through the descrption of the most famous implementations from which many others derive.


\section{Discovery of NoSQL technologies}
\label{sec:context}

Commonly students have their first encounters with database technologies during their studies in high school or bachelor degree and, to better understand all the fundamentals concepts, they are taught the basic principles of relational databases.
It i’s the most common choice of every school teaching the very foundation of Databases to make students understand the the meaning of \textit{CRUD operations, relations, consistency, redundancy}, etc... and how to correctly set up the entities of their systems following proven patterns and constraints.
Detaching from well-known developing habits is not always so simple, but it is necessary to understand why big companies such as Facebook decided to invest money in developing their own database solution, Cassandra, instead of using an existing relational database.
It is important to know that there are many different ways to build a database, some are better than others in certain use cases. Nowadays, an huge amount of data are spread around the world everyday through the Web and it needs to be stored and retriewed quickly to save companies' money and give the users a perfect feeling of resposivity.
But let’s start from the beginning to get an overview of a technology we rely on every day, even without being aware of its presence in every single application we use.


\cite{ictbusiness}
\cite{donoho}

\section{Databases}
\label{sec:problem}

A Database is an organized collection of data even though we often use the term to refer to the entire database system. The Database Management System, or DBMS, is the name of the entire system that handles data, transactions, relations and eventually problems.


\subsection{Relational Databases}
\label{sec:description}

Probably the most popular and for many years also most used model, a relational database is composed by tables representing entities (users, customers, courses...) where each column represents a field or attribute and each row represents a record.
Tables can have relations each other with the use of foreign keys, and each table has a primary key that is unique on each record.


\subsection{Navigational Databases}

\subsection{Object-oriented Databases}

\subsection{NoSQL Databases}

\subsection{The CAP Theorem}

\section {NoSQL: a brief panoramic over the actual situation}

\subsection{MongoDB}

\subsection{Apache Cassandra}

\subsection{Google Big Table}

\subsection{Amazon DynamoDB}

