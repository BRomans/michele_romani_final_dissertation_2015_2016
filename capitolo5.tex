\chapter{Conclusion}
\label{cha:conclusion}
After this evaluation and the presentation of the results, Mongo was confirmed for the new commission. In particular, some considerations had a great impact on the decision: Mongo's ability to support and scan a huge number of records under stress, the low average time needed to retrieve data , the optimal failure rate of 0\% in every test and the simplicity of usage.
There are other NoSQL solutions that state better results in benchmarks, anyway Mongo gave us a sensation of strenght and soundess without saving on performance.
For the future commission the storage architecture will possibily be supported by an SQL solution to store sensible data of the users of the application, while Mongo will be used as storage for all documents related to each user that will be in the grade of billions in the production deployment.
We can now summarize this research and point out some considerations focusing on this question: “how and where NoSQL databases are getting over Relational databases?”.
Looking at NoSQL brief history it seems clear that this technology was born as niche product developed by big IT companiest to support their needs. We can find a huge number of NoSQL databases on the internet all deriving their concepts and implementations from the very first and most famous of them: Amazon Dynamo, Apache Cassandra, Google Big Table and MongoDB.
In the very following years after the concept of NoSQL was born, pretty much every IT company working with Big Data was developing its own version from those models, following specific “needs” but also “fashion”.
SQL and RDBMS will not disappear as they have been a standard for over 30 years and in many use cases where the amount of data is in the grade of Gigabytes, RDBMS  have good performance sometimes even better than many NoSQL databases and will always guarantee ACID propertis that are fundamental in many use cases
It is important to underline that a developer expert of SQL will be an expert user of any kind of RDBMS, while a developer with experience on MongoDB will probably need to learn Apache Cassandra from the very beginning because there is no real standard in NoSQL, each one of them have a proprietary implementation.
But over this consideration and over the performance consideratio, NoSQL technologies have some advantages that will always grant them a step over Relational databases:
\begin{itemize}
	\item Most of them have an open source edition maintained by their companies.
	\item Any developer of the most used programming language such as C++, Java, Python, Perl, C\#, etc. will have no real problem to learn and interface with a NoSQL database.
	\item Most of them are easy to scale and easy to maintain, with consequent saving on time and costs.
	\item They are the only who can support Big Data, a reality of the market that cannot be ignored.
\end{itemize}
Because of those reasons, now NoSQL databases are relegating Relational Databases to more niche roles and getting over as common choice for a storage technology.
In my personal considerations, SQL and NoSQL technologies will keep working in parallel for long time as Specialization is what makes each part of a system obtain the best result for the system itself, and some choices are better than others in specific roles.
Many computer scientists have already started conceiving the concept of \textit{NewSQL} \cite{corbellini}, and some databases have already emerged under the name of “NewSQL databases”, for example NuoDB, VoltDB and Clustrix.
In fact, those databases are actually Relational databases supporting automatic replication, sharding and distributed transactions, i.e. providing ACID guarantees even across shards. 
It is hard to establish what will be the future of storage technologies but in the following years any kind of Databases-related course should be ready to teach students a new way of thinking how to design a database among the classic concepts inherited from the Relational Model.
In this way, next generation of developers will be ready for the new challenge  that NoSQL technologies have opened to our world in terms of storing, manipulating and retrieving data to support the increasing amount of information exchanged through the Web.

