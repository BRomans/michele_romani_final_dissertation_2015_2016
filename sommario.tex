\chapter*{Summary} % senza numerazione
\label{Summary}

\addcontentsline{toc}{chapter}{Summary} % da aggiungere comunque all'indice

The IT world is evolving faster and faster every year, with new breaking technologies coming to our lives, even changing our way to communicate and live in our society. With the advent of social networks, cloud storage and computing, a new definition for the amount of data they involve has been coined: Big Data.
The challenge of Big Data involves both developing better retrieving solutions using advanced data mining techniques and functional storage solutions.
Several companies are switching their old systems and technologies to more scalable and reliable solutions to optimize their costs in terms of time and money, improving their profits.
The company in which I am actually working entrusted me to develop a software in order to evaluate MongoDB, a new non-relational database technology in  anticipation of a new  contract from a customer that needs to support an application with several hundred thousand of users and millions of records.
The challenge is to obtain acceptable results from Mongo in stressing conditions like a production software: retrieving data in less than 2 seconds, preventing loss of data and most importantly, preventing a system crash of the database.
I entirely developed a Java software based on the Spring framework, following my project leader and my tutor directives, capable of launching specific benchmark tests aimed at stress-testing and maybe even crashing a virtual machine running a MongoDB instance.
For my architecture used the technique of Microservices, that consists in building a modular application, with each module dedicated to a specific service. It is an advanced development technique that is getting more and more successful, also thanks to famous use cases such as Netflix, with the strength of easy reusability and maintainability of the software.
My choice of this technique is due to a possible future experimentation of other storage technologies, even relational, as the modularity of the applications allow to quickly develop and connect a new module with drivers for other Database Management Systems.
The choice of MongoDB was made by both our manager and our customer because of its ease of configuration and its availability as 
an open-source software.

%\begin{itemize}
%  \item contesto e motivazioni 
 % \item breve riassunto del problema affrontato
 % \item tecniche utilizzate e/o sviluppate
  %\item risultati raggiunti, sottolineando il contributo personale del laureando/a
%\end{itemize}




